\documentclass[tikz,a4paper,12pt]{article}
%\usepackage{showframe}
\usepackage[utf8x]{inputenc}
\usepackage[margin=2.5cm, bottom=2cm]{geometry}
\setlength{\footskip}{0em}
\usepackage[T1]{fontenc}
\usepackage{ucs}
\usepackage[encapsulated]{CJK}
\usepackage[overlap, CJK]{ruby}
%\usepackage{CJKulem}
\newenvironment{SChinese}{%
  \CJKfamily{gbsn}}{}
\newenvironment{TChinese}{%
  \CJKfamily{bsmi}}{}
%\usepackage{fontspec}
\usepackage{lmodern}
\usepackage{libertine}
\makeatletter
\newcommand\libertineTabular{\def\libertine@figurealign{T}\libertineLF}
\makeatother
\usepackage[british]{babel}
\usepackage[babel=true]{microtype}
\usepackage{natbib}
\setlength{\bibsep}{0.0pt}
\makeatletter
\def\@biblabel#1{#1.}
\makeatother
\usepackage[
  pdfauthor={Paul Sladen, Nineteen Inch Networks},
  pdftitle={Briefing Note on the Inmarsat Publication of 23/24 May 2014 in Relation to the Occurrence of $9$M--MRO},
  pdfsubject={Occurence of 9M-MRO on 7--8 March 2014},
  pdfkeywords={9M-MRO; MH370; MAS370; Flight safety; Inmarsat; ACARS; OOOI; SATCOM; Satellite},
  pdfencoding=unicode,
  backref=section
]{hyperref}
%\usepackage[figure, figure*, table]{hypcap}
%\usepackage[acronym,hyperfirst=false]{glossaries}
\usepackage[figure, figure*]{hypcap}
\usepackage[acronym,hyperfirst=true]{glossaries}
%\usepackage{glossaries-accsupp}
\usepackage[parfill]{parskip}
\usepackage{enumerate}
\usepackage{extdash}
\usepackage{float}
\restylefloat{table}
\usepackage{dcolumn}
\usepackage{siunitx}
\usepackage{booktabs}
\usepackage{calc}
\usepackage{listings}
\usepackage{textcomp}
\usepackage{accsupp}
\newcommand{\nosearch}[1]{\BeginAccSupp{method=hex,unicode,ActualText={}}{#1}\EndAccSupp{}}
\usepackage{textcomp}
\usepackage{pdfpages}
\usepackage{graphicx}
\setlength{\fboxsep}{0.25mm}%
\setlength{\fboxrule}{0.2mm}%
\usepackage{xcolor}
\usepackage{soul,color}
\usepackage{transparent}
\def\greypercent{0.5}
\definecolor{headergrey}{rgb}{\greypercent,\greypercent,\greypercent}
\definecolor{lightyellow}{rgb}{1.0,1.0,0.9}
%\definecolor{lightskyblue}{rgb}{0.53,0.80,0.98}
% [ord(i)/255.0 for i in list(binascii.unhexlify('87cefa'))]
\definecolor{lightskyblue}{rgb}{0.8,0.90,1.0}

\usepackage{fancyhdr}
\fancyfoot{}
\fancypagestyle{plain}{%
  \fancyhf{}
  \renewcommand{\headrulewidth}{0pt}
  \renewcommand{\footrulewidth}{0pt}
  \fancyhead[R]{\nosearch{\textcolor{headergrey}{\textbf{SLA\,{\Large\thepage}}}\hspace*{-2em}}}
  \fancyhead[C]{\textcolor{headergrey}{Briefing note}}
  \fancyhead[L]{\vspace*{-3cm}\makebox[1em]{\nosearch{\transparent{0.5}\includegraphics[keepaspectratio,scale=0.9]{../tex/19inch-logo-grey100.pdf}}}}
  \setlength{\headheight}{1em}
}
\fancyhf{}
\renewcommand{\headrulewidth}{0pt}
\renewcommand{\footrulewidth}{0pt}
\fancyhead[R]{\nosearch{\textcolor{headergrey}{\textbf{SLA\,{\Large{\thepage}}}}\hspace*{-2em}}}
\fancyhead[L]{\hspace*{-2em}\nosearch{\textcolor{headergrey}{Sladen (2014)}}}
%\fancyhead[C]{\hspace*{-2em}\nosearch{\textcolor{headergrey}{Briefing note for AAIB on Inmarsat publication}}}
\fancyhead[C]{\hspace*{-2em}\nosearch{\textcolor{headergrey}{Briefing note for AAIB: Inmarsat publication}}}
%\fancyhead[L]{\vspace*{-3cm}\makebox[1em]{\nosearch{\transparent{0.5}{\includegraphics[keepaspectratio,scale=0.9]{19inch-logo-grey100.pdf}}}}}
\pagestyle{fancy}

% http://www.archivum.info/comp.text.tex/2005-07/00045/Re-Redacted-text-environment-macro-for-LaTeX.html
\sethlcolor{background}
\makeatletter
\def\phantom@SOUL@ulunderline#1{{%
    \setbox\z@\hbox{#1}%
    \dimen@=\wd\z@
    \dimen@i=\SOUL@uloverlap
    \advance\dimen@2\dimen@i
    \rlap{%
        \null
        \kern-\dimen@i
        \SOUL@ulcolor{\SOUL@ulleaders\hskip\dimen@}%
    }%
    \phantom{\unhcopy\z@}% \phantom added here
}}
\DeclareRobustCommand\redact[1]{\begingroup
  \let\SOUL@ulunderline\phantom@SOUL@ulunderline
  \hl{#1}%
\endgroup}
\makeatother
% http://tex.stackexchange.com/questions/83085/how-to-improve-listings-display-of-json-files#answer-83100
\colorlet{punct}{red!60!black}
\definecolor{background}{HTML}{EEEEEE}
\definecolor{delim}{RGB}{20,105,176}
\definecolor{light-gray}{gray}{0.95}
\definecolor{dark-red}{rgb}{0.4,0,0}
\colorlet{numb}{magenta!60!black}
\lstdefinelanguage{json}{
    basicstyle=\tiny\ttfamily,
    numbers=none,
    numberstyle=\tiny,
    stepnumber=1,
    columns=flexible,
    showstringspaces=false,
    numbersep=8pt,
    breaklines=false,
    frame=single,
    backgroundcolor=\color{white},
    morestring=[b]",
    keywords={{=},null},
    resetmargins=true,
    upquote=true,
    literate=
     *{0}{{{\color{numb}0}}}{1}
% Hack to get copy-and-pasteable text
      {\ }{\BeginAccSupp{method=hex,unicode,ActualText=0020}{\color{white}\textvisiblespace}\EndAccSupp{}}{1}
      {\ \ }{\BeginAccSupp{method=hex,unicode,ActualText=00200020}{\color{white}\textvisiblespace\textvisiblespace}\EndAccSupp{}}{1}
      {1}{{{\color{numb}1}}}{1}
      {2}{{{\color{numb}2}}}{1}
      {3}{{{\color{numb}3}}}{1}
      {4}{{{\color{numb}4}}}{1}
      {5}{{{\color{numb}5}}}{1}
      {6}{{{\color{numb}6}}}{1}
      {7}{{{\color{numb}7}}}{1}
      {8}{{{\color{numb}8}}}{1}
      {9}{{{\color{numb}9}}}{1}
      {:}{{{\color{punct}{:}}}}{1}
      {,}{{{\color{punct}{,}}}}{1}
      {\{}{{{\color{delim}{\{}}}}{1}
      {\}}{{{\color{delim}{\}}}}}{1}
      {[}{{{\color{delim}{[}}}}{1}
      {]}{{{\color{delim}{]}}}}{1}
}
\lstdefinelanguage{ACARS}{
    basicstyle=\ttfamily,
    escapechar=§,
    numbers=none,
    numberstyle=,
    stepnumber=1,
    columns=flexible,
    showstringspaces=true,
    numbersep=8pt,
    breaklines=false,
    frame=lines,
    backgroundcolor=\color{lightyellow},
    morestring=[b]",
    keywords={{=},null},
    framexleftmargin=0.5em,framexrightmargin=0.5em,
    resetmargins=true,
    upquote=true,
    literate=
     *{0}{{{\color{dark-red}0}}}{1}
% Hack to get copy-and-pasteable text
      {\ }{\BeginAccSupp{method=hex,unicode,ActualText=0020}{\color{white}\textvisiblespace}\EndAccSupp{}}{1}
      {\ \ }{\BeginAccSupp{method=hex,unicode,ActualText=00200020}{\color{white}\textvisiblespace\textvisiblespace}\EndAccSupp{}}{1}
      {1}{{{\color{dark-red}1}}}{1}
      {2}{{{\color{dark-red}2}}}{1}
      {3}{{{\color{dark-red}3}}}{1}
      {4}{{{\color{dark-red}4}}}{1}
      {5}{{{\color{dark-red}5}}}{1}
      {6}{{{\color{dark-red}6}}}{1}
      {7}{{{\color{dark-red}7}}}{1}
      {8}{{{\color{dark-red}8}}}{1}
      {9}{{{\color{dark-red}9}}}{1}
      {:}{{{\color{punct}{:}}}}{1}
      {,}{{{\color{punct}{,}}}}{1}
      {\{}{{{\color{delim}{\{}}}}{1}
      {\}}{{{\color{delim}{\}}}}}{1}
      {[}{{{\color{delim}{[}}}}{1}
      {]}{{{\color{delim}{]}}}}{1}
}
\lstdefinelanguage{SATCOM}{
    basicstyle=\ttfamily,
    escapechar=§,
    numbers=none,
    numberstyle=,
    stepnumber=1,
    columns=flexible,
    showstringspaces=true,
    numbersep=1.75em,
    breaklines=false,
    frame=lines,
    backgroundcolor=\color{lightyellow},
    morestring=[b]",
    keywords={{=},null},
    framexleftmargin=0.5em,framexrightmargin=0.5em,
    resetmargins=true,
%,xleftmargin=10.75em + 19em width
%,xleftmargin=8.25em + 19em width
%    xleftmargin=8.25em,
%    linewidth=27.25em,
    xleftmargin=8.5em,
    linewidth=27.5em,
    upquote=true,
    literate=
     *{0}{{{\color{dark-red}0}}}{1}
% Hack to get copy-and-pasteable text
      {\ }{\BeginAccSupp{method=hex,unicode,ActualText=0020}{\color{white}\textvisiblespace}\EndAccSupp{}}{1}
      {\ \ }{\BeginAccSupp{method=hex,unicode,ActualText=00200020}{\color{white}\textvisiblespace\textvisiblespace}\EndAccSupp{}}{1}
      {1}{{{\color{dark-red}1}}}{1}
      {2}{{{\color{dark-red}2}}}{1}
      {3}{{{\color{dark-red}3}}}{1}
      {4}{{{\color{dark-red}4}}}{1}
      {5}{{{\color{dark-red}5}}}{1}
      {6}{{{\color{dark-red}6}}}{1}
      {7}{{{\color{dark-red}7}}}{1}
      {8}{{{\color{dark-red}8}}}{1}
      {9}{{{\color{dark-red}9}}}{1}
      {:}{{{\color{punct}{:}}}}{1}
      {,}{{{\color{punct}{,}}}}{1}
      {\{}{{{\color{delim}{\{}}}}{1}
      {\}}{{{\color{delim}{\}}}}}{1}
      {[}{{{\color{delim}{[}}}}{1}
      {]}{{{\color{delim}{]}}}}{1}
}
\DeclareUnicodeCharacter{0100}{\a={A}}
\DeclareUnicodeCharacter{0101}{\a={a}} 
\DeclareUnicodeCharacter{2401}{SOH} 
\setcitestyle{super,open={[},close={]},color=[rgb]{0.46, 0.13, 0.20}}
\hypersetup{colorlinks, urlcolor=[rgb]{0.023, 0.27, 0.68}, linkcolor=[rgb]{0.46, 0.13, 0.20}, citecolor=[rgb]{0.023, 0.27, 0.68}}
% http://tex.stackexchange.com/questions/120644/combined-list-of-figures-and-tables
\makeatletter
\renewcommand*{\ext@figure}{lot}
\let\c@figure\c@table
\let\ftype@figure\ftype@table
\let\listoftableandfigures\listoftables
\renewcommand*\listtablename{List of Tables and figures}
\makeatother

%\raggedright
\mathchardef\mhyphen="2D
\newacronym{MRO}{$9$M--MRO}{Malaysian Boeing~777-2H6/ER aircraft cn.\,28420}
\newacronym{RTT}{RTT}{Round Trip Time}
\newacronym{RoC}{RoC}{Rate of Climb}
\newacronym{ULB}{ULB}{\href{https://en.wikipedia.org/wiki/Underwater_locator_beacon}{Underwater Locator Beacon}}
\newacronym{AAIB}{AAIB}{Air Accident Investigation Branch}
\newacronym{GES}{GES}{Ground Earth Station}
\newacronym{GES301}{GES~301}{\acrlong{GES}~$301_8$ at Eik}
\newacronym{GES305}{GES~305}{\acrlong{GES}~$305_8$ at Perth}
\newacronym{AES}{AES}{Aeronautical Earth Station}
\newacronym{CSV}{CSV}{\href{https://en.wikipedia.org/wiki/Comma-separated_values}{Comma\Hyphdash Separated Value}}
\newacronym{IQ}{I/Q}{\href{https://en.wikipedia.org/wiki/In-phase_and_quadrature_components}{In-phase/Quadrature}}
\newacronym{SDR}{SDR}{\href{https://en.wikipedia.org/wiki/Software-defined_radio}{Software-Defined Radio}}
\newacronym{GEO}{GEO}{\href{https://en.wikipedia.org/wiki/Geostationary_orbit}{Geostationary Orbit}}
\newacronym{c}{$c$}{\href{https://en.wikipedia.org/wiki/Speed_of_light}{speed of light}}
\newacronym{ADS-B}{ADS--B}{\href{https://en.wikipedia.org/wiki/Automatic_dependent_surveillance-broadcast}{Automatic Dependent Surveillance--Broadcast}}
\newacronym{JSON}{JSON}{\href{https://en.wikipedia.org/wiki/JSON}{JavaScript Object Notation}}
%\newacronym{3F1}{1996-020A}{\href{https://en.wikipedia.org/wiki/International_Designator}{internationally designated} 1996-020A/Inmarsat 3-F1/NORAD 23839}
\newacronym{3F1}{$\mathtt{1996\mhyphen 020A}$}{Inmarsat 3-F1 spacecraft}
\newacronym{LNAV}{LNAV}{\href{https://en.wikipedia.org/wiki/LNAV}{Lateral Navigation}}
\newacronym{BFO}{BFO}{Burst Frequency-Offset}
\newacronym{BTO}{BTO}{Burst Timing-Offset}
\newacronym{C-Channel}{C--Channel}{\href{https://en.wikipedia.org/wiki/Circuit_switching}{Circuit-switched} Channel}
\newacronym{R-Channel}{R--Channel}{Random-access \href{https://en.wikipedia.org/wiki/Slotted_ALOHA}{slotted ALOHA} Channel}
\newacronym{T-Channel}{T--Channel}{[Reservation] \href{https://en.wikipedia.org/wiki/Time_division_multiple_access}{Time Division Multiple Access} Channel}
\newacronym{P-Channel}{P--Channel}{Packet-mode \href{https://en.wikipedia.org/wiki/Time-division_multiplexing}{Time Division Multiplex} Channel}
\newacronym{AFC}{AFC}{Automatic Frequency Control}
\newacronym{L-Band}{L--Band}{IEEE L--Band}
\newacronym{C-Band}{C--Band}{IEEE C--Band}
\newacronym{AMSS}{AMS(R)S}{Aeronautical Mobile Satellite (on-Route) Service}
\newacronym{HA01}{HA$01$}{Hydro-Acoustic station~01 (Cape Leeuwin, Australia)}
\newacronym{HA08S}{HA$08$S}{Hydro-Acoustic station~08 South (British Indian Ocean Territory)}
\newacronym{RCS}{RCS}{Hydrophone bouy off Rottnest Island, Perth}
\newacronym{IMOS}{IMOS}{Integrated Marine Observing System}
\newacronym{CTBTO}{CTBTO}{\href{https://en.wikipedia.org/wiki/Comprehensive_Nuclear-Test-Ban_Treaty_Organization}{Comprehensive Nuclear-Test-Ban Treaty Organization}}
\newacronym{GSM}{GSM}{GSM}
\newacronym{CDMA}{CDMA}{Code Division Multiple Access}
\newacronym{ICAO}{ICAO}{International Civil Aviation Organization}
\newacronym{FDR}{FDR}{Flight Data Recorder}
\newacronym{CVR}{CVR}{Cockpit Voice Recorder}
\newacronym{75008F}{$\mathtt{75008F}_{16}$}{\acrlong{ICAO} 24-bit address $\mathtt{75008F}_{16}$}
\newacronym{35200217}{$\mathtt{35200217}_{8}$}{\acrlong{ICAO} 24-bit address $\mathtt{35200217}_{8}$}
\newacronym[sort=GES C5]{C5}{$\mathtt{C5}_{16}$}{GES $\mathtt{C5}_{16}$ in Perth}
\newacronym{GESC5}{GES~$\mathtt{C5}_{16}$}{GES~$\mathtt{C5}_{16}$ in Perth}
\newacronym[sort=Radar]{RADAR}{\textsc{Radar}}{R\textsc{a}dio Detection And Ranging}
\newacronym{ACARS}{ACARS}{Aircraft Communications Addressing and Reporting System}
\newacronym{MCDU}{MCDU}{Multifunction Control Display Unit}
\newacronym{OOOI}{OOOI}{Out~Off~On~In}
\newacronym{FMC}{FMC}{Flight Management Computer}
\newacronym{SAT}{SAT}{Satellite}
\newacronym{CRC16}{CRC16}{16-bit Cyclic Redundancy Check}
\newacronym{SDU}{SDU}{Satellite Data Unit}
\newacronym{RLS}{RLS}{Reliable Link Service}
\newacronym{SU}{SU}{Signalling Unit}
\newacronym{MDCA}{MDCA}{Malayasian Department of Civil Aviation}
\newacronym{GCHQ}{GCHQ}{Government Communications Headquarters}
\newacronym{IATA}{IATA}{International Air Transport Association}
\newacronym{ISU}{ISU}{Initial Signalling Unit}
\glossarystyle{long}
\renewcommand{\glsgroupskip}{}%
\setlength\glsdescwidth{\linewidth}%
\makeglossaries
\usepackage{microtype}
\makeatletter
%non-file image placeholders
%\def\Ginclude@graphics#1{%
%  \begingroup\fboxsep=-\fboxrule
%  \framebox[\@ifundefined{Gin@@ewidth}{150pt}{\Gin@@ewidth}]%
%      {\rule{0pt}{\@ifundefined{Gin@@eheight}{100pt}{\Gin@@eheight}}
%      \raisebox{3pt}{\texttt{#1}}}%
%  \endgroup}
%\makeatother 

\newcommand{\numold}[1]{#1}
% Allow ``No Starch Press''-like custom line numbers (essentially, bulleted line numbers for only those lines the author will address)
\newcounter{lstNoteCounter}
\newcommand{\lnnum}[1]{#1}
\newcommand*{\lnote}{\stepcounter{lstNoteCounter}\vbox{\llap{{\lnnum{\thelstNoteCounter}}\hskip 1em}}}
\lstnewenvironment{csource2}[1][]
{
    \setcounter{lstNoteCounter}{0}
    \lstset{basicstyle=\ttfamily,numberstyle=\numold,numbers=right,frame=lines,framexleftmargin=0.5em,framexrightmargin=0.5em,backgroundcolor=\color{lightyellow},showstringspaces=true,escapeinside={(*@}{@*)},#1}
}
{}
%\usepackage[abs]{overpic}
%\setlength\unitlength{0.1mm}
%\usepackage[pdf]{pstricks}
%\usepackage[on]{auto-pst-pdf} 
\usepackage{tikz}
\usepackage{picture}
\usepackage{adjustbox}
\usepackage{footmisc}
\begin{document}
\author{\href{http://www.paul.sladen.org/}{Paul Sladen%
},\, \href{http://www.19inch.net/}{Nineteen Inch Networks},\, Nottingham GBR
}
\title{%
Briefing note on the Inmarsat publication of 23--24~May~2014
pursuant to the occurrence of
$9$M\BeginAccSupp{method=hex,unicode,ActualText=002D}--\EndAccSupp{}MRO~on~7--8~March~2014%
}
\date{
%Unc%
C%
irculated to the \acrlong{AAIB}
%as of
on
3~July~2014\\\textsc{in re} \href{https://www.whatdotheyknow.com/request/9m_mro75008f_aes_exchange_iq_sni}{\BeginAccSupp{method=hex,unicode,ActualText=0050}\textsc{p\hspace{0.09em}}\EndAccSupp{}0011293}}
\maketitle
\vspace{-0.7em}
%\bigskip
%\section*{Abstract}

{This briefing note offers a non-technical overview of Inmarsat logs
published on 23/24~May~2014 and how these relate to a later
redacted publication by the Malayasian DCA\footnote{\acrlong{MDCA}.\,\href{http://www.dca.gov.my/mainpage/MH370\%20Data\%20Communication\%20Logs.pdf}{www.dca.gov.my/mainpage/MH370~Data~Communication~Logs.pdf}} on 27~May~2014.\unskip\parfillskip-0.103em \par}
\vspace{-1.5em}
\section{Introduction}
\vspace{-0.70em}
{On 20/23~May 2014 CNN recorded television footage at Inmarsat HQ in
London.  This was broadcast on 24~May 2014 and subsequently screened
as three cuts of varying length.%
\footnote{CNN broadcast lengths (and 1280$\times$720 archive filesizes):\hspace{0.4em}
\href{http://cnn-f.akamaihd.net/cnn/big/world/2014/05/27/intv-quest-inmarsat-data-dickinson.cnn_27014108_,512x288_55,640x360_90,768x432_130,896x504_185,1280x720_350,0k.mp4.csmil/bitrate=4}
{3m\,34s} (74\,\textsc{MB}),\hspace{0.4em}
\href{http://cnn-f.akamaihd.net/cnn/big/world/2014/05/27/pkg-quest-inmarsat-data.cnn_27020448_,512x288_55,640x360_90,768x432_130,896x504_185,1280x720_350,0k.mp4.csmil/bitrate=4}
{4m\,54s} (101\,\textsc{MB}),\hspace{0.4em}
\href{http://cnn-f.akamaihd.net/cnn/big/world/2014/05/27/intv-quest-uk-inmarsat-dickinson-long.cnn_27080346_,512x288_55,640x360_90,768x432_130,896x504_185,1280x720_350,0k.mp4.csmil/bitrate=4}
{9m\,46s} (201\,MB)}
The bulk of material came from a one-to-one interview between CNN
reporter Richard Quest and Inmarsat Director of Satellite Operations
Mark Dickinson during the morning of
23~May~2014.\unskip\parfillskip-0.103em \par}

{During this interview Dickinson openly displayed several pages of printed log
files.  These sections form the unredacted records of communication between
Inmarsat's ground infrastructure and the \gls{MRO} for 7~March~2014.\unskip\parfillskip-0.103em \par}

\newcommand\measurepage{\dimexpr\pagegoal-\pagetotal-\baselineskip\relax}

%\vfill
%\vspace{-0.25em}
\vspace{0.25em}
\begin{figure}[ht!]
\label{cnn-dickinson-caption}
\setlength{\abovecaptionskip}{-0.2em plus 0.5em minus 0.5em}
%\renewcommand{\textfraction}{10}
%\setlength{\floatsep}{5pt plus 2pt minus 2pt}
%\setlength{\textfloatsep}{5pt plus 2pt minus 2pt}
%\setlength{\intextsep}{5pt plus 2pt minus 2pt}
\centering
%\includegraphics[width=0.84\textwidth]{cnn-dickinson-quest-caption-1280x696.jpg}
%\includegraphics[height\textheight]{cnn-dickinson-quest-caption-1280x696.jpg}
%\includegraphics[height=\measurepage-1.80em,keepaspectratio]{cnn-dickinson-quest-caption-1280x696.jpg}
%\includegraphics[width=0.94\textwidth,height=\measurepage-3em,keepaspectratio,trim=0 10mm 0 0,clip]{cnn-dickinson-quest-holding-caption-1280x720.jpg}
\resizebox{0.94\textwidth}{!}{%
% left bottom right top
\clipbox{0 2.55mm 0 0}{%
\begin{picture}(128.0mm, 72.0mm)(0,0)%
\put(0,0){\includegraphics[width=128.0mm,keepaspectratio]{cnn-dickinson-quest-holding-caption-1280x720.jpg}}%
\put(0,0){\includegraphics[width=128.0mm,keepaspectratio]{{dickinson-quest-overlay}.pdf}}%
\end{picture}%
}}%
\caption[Mark Dickinson and Richard Quest with logs at Inmarsat HQ---23 May 2014]{Reviewing a page of the unredacted signalling unit log files in London at Inmarsat~HQ on 23~May~2014.
\textsc{Left:} Mark Dickinson, Inmarsat. \textsc{Right:} Richard Quest, CNN\hfill(CNN)}
\end{figure}
\vspace{-0.25em}


\newpage

\subsection{Radio frequencies}

Boeing 777 aircraft are equiped with several radio-frequency
communication systems.  Amongst these are Airband VHF radio,
long-distance HF radio and L-Band satellite.  VHF Airband works
immediately above the frequencies used for FM~broadcasting and is at
108--137\,MHz.  HF (high-frequency) can travel longer distances because
it uses much lower frequencies than VHF communication. Conversely,
satellite Aero L-Band uses higher frequencies than VHF, receiving at
1545--1555\,MHz and transmitting at 1646.5--1656.5\,MHz.  All three sets of
frequencies can transport voice and simple Telex messaging/data support, at a
level comparable to 1990s-era GSM mobile telephones.

The L-Band frequencies are close to those used by GPS, and so only
work outdoors where there is direct line-of-sight to a satellite.  Some
other automatically-activated radios use the frequencies in-between:
secondary \acrshort{RADAR} responders at 1030\,MHz/1090\,MHz, and Emergency Locator Beacons
(ELTs) at 406\,MHz.  We will only cover the Aero satellite radio communications.

\subsection{ACARS and SATCOM}

The \gls{ACARS} performs transhipment of short messages and
data fragments over the VHF, HF or Satcom links.  At the other
end of the links the \gls{ACARS} network providers (a bit like an ISP)
ensure the messages are transported onwards and delivered to each
airline's operations centre.  The dominant ACARS providers are SITA in
Asia, and ARNIC in the Americas.

On the aircraft side, interaction with the ACARS applications, and configuration of
Satcom takes place via two tablet-sized \glsplural{MCDU}
in the centre pedestal of the cockpit. (Figure~\ref{9m-mro-cockpit})

\begin{figure}[ht!]
\centering
\label{9m-mro-cockpit}
\setlength{\abovecaptionskip}{-0.2em plus 0.5em minus 0.5em}
\resizebox{0.94\textwidth}{!}{
%36 38 27 32
%452mm-101mm-94mm,289mm-78mm-75mm
%\clipbox{101mm 78mm 94mm 75mm}{
% left bottom right top
\clipbox{37.5mm 30mm 35mm 29mm}{
\begin{picture}(160mm, 102.3mm)(0,0)
\put(0,0){\includegraphics[angle=-0.49,width=160mm,keepaspectratio]{20040422-9M-MRO-cockpit-Chris-Finney-GFDL12only-0561319-lossless.pdf}}%
\put(0,0){\includegraphics[angle=-0.49,width=160mm,keepaspectratio]{{cockpit-overlay-left-right-only}.pdf}}
\end{picture}
}}
\caption[$9$M-MRO centre console---22 April 2004]{$9$M-MRO cockpit.
\textsc{Bottom:} Satcom Main Menu displayed on right-hand \acrlong{MCDU}.
\textsc{Middle:} SELCALL/Registration reminder stickers.
\hfill(Chris Finney\, \textsc{gfdl\,1.2})}
\end{figure}

The \acrlong{MCDU} acts in a similiar manner to a mobile phone.
Several applications (eg. Weather, Diagnostics, Flight Management) can
all be used from the same screen and keyboard.  Some applications will
only activate when the user chooses them, while others (eg. Position reporting, \gls{OOOI},
Engine Health) will normally run autonomously in
the background.  In Figure~\ref{9m-mro-cockpit} the \emph{left-hand}
\gls{MCDU} is shown displaying the choices \gls{FMC} and
\gls{SAT}.

\section{ACARS Messages}

Communitions between aircraft and the ground are simple and
text-based.  Like with emails and HTML webpages, the underlying textual
nature can be followed by humans.  The grammar is terse, much akin to
old telegraph and Telex messages, and without the excesses of word
spacing and full-stops.

We will trace a single human-readable textual message, right through
the stages of its preparation for transmission within the \gls{ACARS} system
in the aircraft, out via SATCOM, and until it reaches the \gls{GES}.  We
will then compare this with the published log files that are the
subject of this briefing note.

\subsection{ACARS Out Off On In}

Just after 16:41:39 on 7 March 2014 the event reporter on-board the aircraft requested
to transmit a message showing that the aircraft had taken off from a runway.  This ``sentence'' was:

\begin{lstlisting}[language=ACARS,linewidth=\textwidth,numbers=left]
OFF01MAS370    /--071641WMKKZBAA
1641 492
\end{lstlisting}

The first line shows take-off time, plus departure and destination aerodrome;
while the second line covers fuel availability.  The individual ``words'' in the message are:

\begin{center}
\begin{tabular}{p{1.5cm}p{2cm}p{8cm}}
\hfill\texttt{OFF01}&event&taking \emph{off} (first message)\\
\hfill\texttt{MAS370}&callsign&\acrshort{ICAO} code \emph{MAS}\,+\emph{370} flight number\\
\hfill\texttt{071641}&timestamp& \emph{07} March 2014 \emph{16:41} UTC\\
\hfill\texttt{WMKK}&from&\emph{Kuala Lumpur}\\
\hfill\texttt{ZBAA}&to&\emph{Beijing Capital}\\
\hline
\hfill\texttt{1641}&time&\emph{16:41} UTC\\
\hfill\texttt{492}&fuel&calculated \emph{49,200 kg} fuel on-board (FOB)\\
\end{tabular}
\end{center}

\subsection{ACARS Application}

The \gls{OOOI} event reporter is one of several applications that can run
over the intrastructure provided by \gls{ACARS}.  These applications are
like the many ``apps'' that can be found on a smartphone, each has a
different use but all of then use the same display, keyboard, and network
connection. To ensure that any reply can be routed back to the correct
application on-board the aircraft some additional contextual information needs to be added at the
start:

\begin{lstlisting}[language=ACARS,linewidth=\textwidth]
M02AMH0370§\textcolor{gray}{OFF01MAS370}§    §\textcolor{gray}{/-{}-071641WMKKZBAA}§
§\textcolor{gray}{1641 492}§
\end{lstlisting}

These are a message ID to allow for tracking, and the flight ID so
that it is clear which flight segment this relates to:

\begin{center}
\begin{tabular}{p{1.5cm}p{2cm}p{8cm}}
\hfill\texttt{M02A}&message ID&\emph{Message 02-A} (it can be anything unique)\\
\hfill\texttt{MH0370}&flight ID&\acrshort{IATA} code \emph{MH}\,+\emph{0370} flight number\\
\hline
\hfill\textcolor{gray}{\texttt{\ldots}}&\textcolor{gray}{message}&\textcolor{gray}{[see above]}\\
\end{tabular}
\end{center}

\newpage
\subsection{ACARS Datalink Header}

{Each stage in preparing for transmission wraps the message with some extra information:\unskip\parfillskip-0.103em \par}

\begin{lstlisting}[language=ACARS,numbers=left,linewidth=\textwidth]
2.9M-MRO<NAK>122<STX>M02AMH0370§\textcolor{gray}{OFF01MAS370}§    §\textcolor{gray}{/-{}-071641WMKKZBAA}§\r\n
§\textcolor{gray}{1641 492}§<ETX>
\end{lstlisting}

{With the extra header and footer parts, the main message it is still readable. New parts are:\unskip\parfillskip-0.103em \par}

\begin{center}
\begin{tabular}{p{1.5cm}p{2cm}p{8cm}}
\hfill\texttt{2}&mode&always \emph{2}\\
\hfill\texttt{9M-MRO}&registration&aircraft regisration, \emph{9M-} is Malayasia\\
\hfill\texttt{<NAK>}&acknowledge&\emph{Negative Acknowledgement}\\
\hfill\texttt{12}&label&a type \emph{12} message\\
\hfill\texttt{2}&DBI&always \emph{2}\\
\hfill\texttt{<STX>}&text&\emph{Start of Text} part\\
\hline
\hfill\textcolor{gray}{\texttt{\ldots}}&\textcolor{gray}{message}&\textcolor{gray}{[see above]}\\
\hline
\hfill\texttt{<ETX>}&text&\emph{End of Text} part\\
\end{tabular}
\end{center}

\subsubsection*{Hexadecimal view}
At this point we will introduce a second representation of the same
message, but using the values of the characters as the computer see
them.  These values are shown as pairs of hexadecimal digits (\texttt{0--F}).  We can follow the hex string and pick out the decimal numbers in the
\textcolor{red}{flight} ID, the date/\textcolor{blue}{time}, the
\textcolor{green}{fuel} load, and the repeated characters in the
registration
\textsc{`\textcolor{red}{9}m\textcolor{lightgray}{-}m\textcolor{lightgray}{ro}'},
the word \textsc{`\textcolor{lightgray}{o}ff'}, the two hyphens `-{}-', and
the two airport codes \textsc{`\textcolor{lightgray}{wm}kk'} and
\textsc{`\textcolor{lightgray}{zb}aa'}:

\begin{lstlisting}[language=ACARS]
         §\textcolor{lightgray}{32 2E 3\textcolor{red}{9} \textcolor{black}{4D} 2D \textcolor{black}{4D} 52 4F 15 31 32 32 02 4D 30 32 41 4D 48 3\textcolor{red}{0} 3\textcolor{red}{3}}§
§\textcolor{lightgray}{3\textcolor{red}{7} 3\textcolor{red}{0} 4F \textcolor{black}{46 46} 30 31 4D 41 53 3\textcolor{red}{3} 3\textcolor{red}{7} 3\textcolor{red}{0} 20 20 20 20 2F \textcolor{black}{2D 2D} 3\textcolor{blue}{0} 3\textcolor{blue}{7} 3\textcolor{blue}{1} 3\textcolor{blue}{6}§
§3\textcolor{blue}{4} 3\textcolor{blue}{1}} \textcolor{lightgray}{57 4D \textcolor{black}{4B 4B} 5A 42 \textcolor{black}{41 41} 0D 0A 3\textcolor{blue}{1} 3\textcolor{blue}{6} 3\textcolor{blue}{4} 3\textcolor{blue}{1} 20 3\textcolor{green}{4} 3\textcolor{green}{9} 3\textcolor{green}{2} 03}
\end{lstlisting}

\subsection{ACARS Datalink parity}

Humans work best with letters and words, but the computer is working
with numbers.  So far these values for the characters have all been
limited to the 128 values between 0--127 (\texttt{00--7F}). Adding a
parity bit means there are now 256 possible values between 0--255
(\texttt{00--FF}) and these are called \textbf{octets} because they consist of 8-bits.  These extra values can be used for detecting
errors because, although all of the values \emph{are possible}, not all of the combinations \emph{are allowed:}

\begin{lstlisting}[language=ACARS]
         §\textcolor{lightgray}{32 \textcolor{black}{AE B9 CD AD CD} 52 4F 15 31 32 32 02 \textcolor{black}{CD B0} 32 \textcolor{black}{C1 CD C8 B0 B3}}§
§\textcolor{lightgray}{37 \textcolor{black}{B0} 4F 46 46 \textcolor{black}{B0} 31 \textcolor{black}{CD C1 D3 B3} 37 \textcolor{black}{B0} 20 20 20 20 2F \textcolor{black}{AD AD B0} 37 31 \textcolor{black}{B6}}§
§\textcolor{lightgray}{34 31 57 \textcolor{black}{CD CB CB DA C2 C1 C1} 0D \textcolor{black}{8A} 31 \textcolor{black}{B6} 34 31 20 34 \textcolor{black}{B9} 32 \textcolor{black}{83}}
\end{lstlisting}


Half of the values have remained the same, and half have been altered in a predictable way:

\begin{center}
\begin{tabular}{c|c|cl}
number & encoding & with parity &\\[0.5em]
\texttt{`0'} & \texttt{30} & \textbf{\texttt{B0}}&$\leftarrow$\\
\texttt{`1'} & \texttt{31} & \texttt{31}&\\
\texttt{`2'} & \texttt{32} & \texttt{32}&\\
\texttt{`3'} & \texttt{33} & \textbf{\texttt{B3}}&$\leftarrow$\\
\texttt{`4'} & \texttt{34} & \texttt{34}&\\
\end{tabular}
\quad\quad
\begin{tabular}{c|c|cl}
number & encoding & with parity &\\[0.5em]
\texttt{`5'} & \texttt{35} & \textbf{\texttt{B5}}&$\leftarrow$\\
\texttt{`6'} & \texttt{36} & \textbf{\texttt{B6}}&$\leftarrow$\\
\texttt{`7'} & \texttt{37} & \texttt{37}&\\
\texttt{`8'} & \texttt{38} & \texttt{38}&\\
\texttt{`9'} & \texttt{39} & \textbf{\texttt{B9}}&$\leftarrow$\\
\end{tabular}
\end{center}

\subsection{ACARS Datalink checksum}

The next stage is a checksum, which enables the integrity of the full
message to be validated.  A common everyday example of a checksum is
the last digit on a credit card---which is not part of the number
itself but is an artifact calculated by adding up the previous fifteen
digits.  Including the checksum digit makes it possible to quickly
catch a number which has been read, or transcribed incorrectly.

For the \gls{ACARS} checksum, a \gls{CRC16} is used for producing the summary.  The particular \gls{CRC16} variant used is called ``Kermit''
(named after the frog), and starts with the value zero (\texttt{00~00}), before adding and multiplying the message values together.

\begin{lstlisting}[language=ACARS,numbers=left,linewidth=\textwidth]
<SOH>§\textcolor{gray}{2.9M-MRO<NAK>122<STX>M02AMH0370OFF01MAS370}§    §\textcolor{gray}{/-{}-071641WMKKZBAA\textbackslash r\textbackslash n}§
§\textcolor{gray}{1641 492<ETX>}§8.<DEL>
\end{lstlisting}

\begin{lstlisting}[language=ACARS]
      §\textcolor{lightgray}{\textcolor{black}{01} 32 AE B9 CD AD CD 52 4F 15 31 32 32 02 CD B0 32 C1 CD C8 B\textcolor{red}{0} B\textcolor{red}{3}}§
§\textcolor{lightgray}{3\textcolor{red}{7} B\textcolor{red}{0} 4F 46 46 B0 31 CD C1 D3 B\textcolor{red}{3} 3\textcolor{red}{7} B\textcolor{red}{0} 20 20 20 20 2F AD AD B\textcolor{blue}{0} 3\textcolor{blue}{7} 3\textcolor{blue}{1} B\textcolor{blue}{6}}§
§\textcolor{lightgray}{3\textcolor{blue}{4} 3\textcolor{blue}{1} 57 CD CB CB DA C2 C1 C1 0D 8A 3\textcolor{blue}{1} B\textcolor{blue}{6} 3\textcolor{blue}{4} 3\textcolor{blue}{1} 20 3\textcolor{green}{4} B\textcolor{green}{9} 3\textcolor{green}{2} 83 \textcolor{black}{B8 7E 7F}}§
\end{lstlisting}

\begin{center}
\begin{tabular}{p{1.5cm}p{2cm}p{8cm}}
\hfill\texttt{<SOH>}&header&\emph{Start Of Header}\\
\hline
\hfill\textcolor{gray}{\texttt{\ldots}}&\textcolor{gray}{message}&\textcolor{gray}{[see above]}\\
\hline
\hfill\texttt{B8 7E}&checksum&calculated total of previous octets\\
\hfill\texttt{<DEL>}&footer&\emph{Delete}, end, stop processing
\end{tabular}
\end{center}

\dots

At this point the message is ready and is identical whether sent via
VHF\,Com, via HF\,Com or via Satcom.  The choice for how to send the
message is one of availability verses cost.  VHF is cheap, but only
works near land.  Inmarsat's \emph{Classic Aero} \acrshort{AMSS} service is expensive, but works across the
whole globe with the only exceptions being close to the North and
South poles.

The next step is to hand the complete encapsulated ACARS Datalink
message to the aircraft's \gls{SDU}.

\section{Satcom Data-2}

ACARS messages can be transmitted over the \gls{AMSS}, more commonly shortened to ``Satcom''.
When sending an ACARS messages via Satcom it is first ``enveloped'' but without any address.
This leaves all of the decisions and complexity about where to forward the received ACARS message to
the \gls{GES} .  All the
\acrlong{SDU} does is to add a two-octet header on the
front consisting of \texttt{FF~FF}.
\vfill
\begin{lstlisting}[language=ACARS,linewidth=\textwidth]
..§\textcolor{gray}{<SOH>2.9M-MRO<NAK>122<STX>M02AMH0370OFF01MAS370}§    §\textcolor{gray}{/-{}-071641WMKKZBAA\textbackslash r\textbackslash n}§
§\textcolor{gray}{1641 492<ETX>8.<DEL>}§
\end{lstlisting}

\begin{lstlisting}[language=ACARS]
§\textcolor{lightgray}{\textcolor{black}{FF FF} 01 32 AE B9 CD AD CD 52 4F 15 31 32 32 02 CD B0 32 C1 CD C8 B\textcolor{red}{0} B\textcolor{red}{3}}§
§\textcolor{lightgray}{3\textcolor{red}{7} B\textcolor{red}{0} 4F 46 46 B0 31 CD C1 D3 B\textcolor{red}{3} 3\textcolor{red}{7} B\textcolor{red}{0} 20 20 20 20 2F AD AD B\textcolor{blue}{0} 3\textcolor{blue}{7} 3\textcolor{blue}{1} B\textcolor{blue}{6}}§
§\textcolor{lightgray}{3\textcolor{blue}{4} 3\textcolor{blue}{1} 57 CD CB CB DA C2 C1 C1 0D 8A 3\textcolor{blue}{1} B\textcolor{blue}{6} 3\textcolor{blue}{4} 3\textcolor{blue}{1} 20 3\textcolor{green}{4} B\textcolor{green}{9} 3\textcolor{green}{2} 83 B8 7E 7F}§
\end{lstlisting}

\begin{center}
\begin{tabular}{p{1.5cm}p{2cm}p{8cm}}
\hfill\texttt{FF FF}&envelope&\emph{Arrange automatic ACARS forwarding}\\
\hline
\hfill\textcolor{gray}{\texttt{\ldots}}&\textcolor{gray}{message}&\textcolor{gray}{[see above]}\\
\end{tabular}
\end{center}

\subsection{SATCOM Reliable Link Service}

The \gls{RLS} is a part of the \gls{AMSS} system which ensures that
only whole and complete messages are forwarded on.  The RLS achives
its \emph{reliability} by cooperation of the \gls{AES} on the aircraft and \gls{GES}
to arrange for retransmiting individual missing parts.  The \gls{RLS}
prepares the full message by chopping it up into smaller pieces.  No
new octets are added during the splitting.

{There is space for \textbf{two octets} in the first part, then space for \textbf{eight
octets} in the subsequent parts:\unskip\parfillskip-0.103em \par}\vspace{0.5em}

\begin{lstlisting}[language=ACARS]
§\textcolor{lightgray}{FF FF 01 32 AE B9 CD AD CD 52 4F 15 31 32 32 02 CD B0 32 C1 CD C8 B0 B3}§
§\textcolor{lightgray}{37 B0 4F 46 46 B0 31 CD C1 D3 B3 37 B0 20 20 20 20 2F AD AD B0 37 31 B6}§
§\textcolor{lightgray}{34 31 57 CD CB CB DA C2 C1 C1 0D 8A 31 B6 34 31 20 34 B9 32 83 B8 7E 7F}§
\end{lstlisting}

%\begin{center}\vspace{-0.9em}\hspace{-1.42em}$\Downarrow $\hspace{7em}$\Downarrow $\end{center}
\vspace{-0.75em}\hspace{12.3em}$\Downarrow $\hspace{8.67em}\raisebox{0.8em}{$\Downarrow $}\vspace{0.1em}
\begin{lstlisting}[language=SATCOM,numbers=left]
                         §\textcolor{darkgray}{FF FF}§                 ..
       §\textcolor{darkgray}{01 32 AE B9 CD AD CD 52}§           .2.9M-MR
       §\textcolor{darkgray}{4F 15 31 32 32 02 CD B0}§           O.122.M0
       §\textcolor{darkgray}{32 C1 CD C8 B0 B3 37 B0}§           2AMH0370
       §\textcolor{darkgray}{4F 46 46 B0 31 CD C1 D3}§           OFF01MAS
       §\textcolor{darkgray}{B3 37 B0 20 20 20 20 2F}§           370    /
       §\textcolor{darkgray}{AD AD B0 37 31 B6 34 31}§           --071641
       §\textcolor{darkgray}{57 CD CB CB DA C2 C1 C1}§           WMKKZBAA
       §\textcolor{darkgray}{0D 8A 31 B6 34 31 20 34}§           ..1641 4
       §\textcolor{darkgray}{B9 32 83 B8 7E 7F}§                 92.8..
\end{lstlisting}

Each of the ten parts will become a \textbf{Signalling Unit}.  The
complete set of the units will become a
\textbf{Signalling Unit Set} .

\subsection{SATCOM padding}

All of the parts except for the first must be exactly eight octets in
length.  When the last part of the split message does not align
exactly to the boundary, it must be \emph{padded out} to the full width; zeros (\texttt{00}) are used.  The number of genuine (non-padding) octets
in the last part is recorded.

\vspace{0.5em}

\begin{lstlisting}[language=SATCOM,numbers=none]
                      §\textcolor{black}{6}\textcolor{darkgray}{0} \textcolor{lightgray}{FF FF}§                `..
       §\textcolor{lightgray}{01 32 AE B9 CD AD CD 52}§           .2.9M-MR
       §\textcolor{lightgray}{4F 15 31 32 32 02 CD B0}§           O.122.M0
       §\textcolor{lightgray}{32 C1 CD C8 B0 B3 37 B0}§           2AMH0370
       §\textcolor{lightgray}{4F 46 46 B0 31 CD C1 D3}§           OFF01MAS
       §\textcolor{lightgray}{B3 37 B0 20 20 20 20 2F}§           370    /
       §\textcolor{lightgray}{AD AD B0 37 31 B6 34 31}§           --071641
       §\textcolor{lightgray}{57 CD CB CB DA C2 C1 C1}§           WMKKZBAA
       §\textcolor{lightgray}{0D 8A 31 B6 34 31 20 34}§           ..1641 4
       §\textcolor{lightgray}{\underline{B9 32 83 B8 7E 7F}} \textcolor{black}{00 00}§           92.8....
\end{lstlisting}

%In order to distinguish which are padding octets and which are the
%last part of the message, 

\begin{center}
\begin{tabular}{p{1.5cm}p{2cm}p{8cm}}
\hfill\texttt{6\textcolor{darkgray}{0}}&length&\emph{6} valid octets in last part\\
\hline
\hfill\textcolor{gray}{\texttt{\ldots}}&\textcolor{gray}{message}&\textcolor{gray}{[see above]}\\
\hline
\hfill\texttt{00}&padding&fill to end ($8-6=2\times padding$\,)\\
\end{tabular}
\end{center}

\clearpage
\subsection{SATCOM sequence and priority}

To show that all of the parts belong to the same set, they are each
given a sequence number (position) within the set, plus a shared identifier for the whole
set and a shared prioritisation rating.

\begin{lstlisting}[language=SATCOM,numbers=left]
                §\textcolor{black}{7D} \textcolor{darkgray}{0}§9 §\textcolor{lightgray}{60 FF FF}§              }.`..
 8 §\textcolor{black}{7D}§  §\textcolor{lightgray}{01 32 AE B9 CD AD CD 52}§         .}.2.9M-MR
 7 §\textcolor{black}{7D}§  §\textcolor{lightgray}{4F 15 31 32 32 02 CD B0}§         .}O.122.M0
 6 §\textcolor{black}{7D}§  §\textcolor{lightgray}{32 C1 CD C8 B0 B3 37 B0}§         .}2AMH0370
 5 §\textcolor{black}{7D}§  §\textcolor{lightgray}{4F 46 46 B0 31 CD C1 D3}§         .}OFF01MAS
 4 §\textcolor{black}{7D}§  §\textcolor{lightgray}{B3 37 B0 20 20 20 20 2F}§         .}370    /
 3 §\textcolor{black}{7D}§  §\textcolor{lightgray}{AD AD B0 37 31 B6 34 31}§         .}--071641
 2 §\textcolor{black}{7D}§  §\textcolor{lightgray}{57 CD CB CB DA C2 C1 C1}§         .}WMKKZBAA
 1 §\textcolor{black}{7D}§  §\textcolor{lightgray}{0D 8A 31 B6 34 31 20 34}§         .}..1641 4
 0 §\textcolor{black}{7D}§  §\textcolor{lightgray}{B9 32 83 B8 7E 7F 00 00}§         .}92.8....
\end{lstlisting}

The priority and message set identifier both remain common to all of
the parts, while the sequence number counts down to zero.  A longer
original message will use a starting sequence number which is higher.  The last sequence number will always be zero.

\begin{center}
\begin{tabular}{p{1.5cm}p{2cm}p{8cm}}
\hfill\texttt{9--0}&sequence no&position in message (counting down)\\
\hfill\texttt{D}&grouping&Signalling set identifier (same across the set)\\
\hfill\texttt{7}&priority&\emph{normal} priority (scale of 0--15)\\
\hline
\hfill\textcolor{gray}{\texttt{\ldots}}&\textcolor{gray}{message}&\textcolor{gray}{[split parts from above]}\\
\end{tabular}
\end{center}

\subsection{SATCOM addressing}

Every ACARS message sent over Satcom is re-broadcast to one-third of
the whole the planet.  Because of the number of listeners it is necessary to
know precisely which \gls{GES} the message is destined for, and whom the sender
was.

The identities of the sender and destination are added to the first
part of the set.  The \emph{from address} is the unique address of the
\gls{AES} onboard the aircraft.  For \gls{MRO} this is \acrshort{75008F}.  The \emph{to address} is the unique
address of the \gls{GES} in Perth which is \acrshort{GES305}, or \acrshort{C5} in hexadecimal form:

\begin{lstlisting}[language=SATCOM,numbers=none]
   75  00 8F C5 §\textcolor{lightgray}{7D 09 60 FF FF}§          u...}.`..
 §\textcolor{lightgray}{8 7D}§  §\textcolor{lightgray}{01 32 AE B9 CD AD CD 52}§         .}.2.9M-MR
 §\textcolor{lightgray}{7 7D}§  §\textcolor{lightgray}{4F 15 31 32 32 02 CD B0}§         .}O.122.M0
 §\textcolor{lightgray}{6 7D}§  §\textcolor{lightgray}{32 C1 CD C8 B0 B3 37 B0}§         .}2AMH0370
 §\textcolor{lightgray}{5 7D}§  §\textcolor{lightgray}{4F 46 46 B0 31 CD C1 D3}§         .}OFF01MAS
 §\textcolor{lightgray}{4 7D}§  §\textcolor{lightgray}{B3 37 B0 20 20 20 20 2F}§         .}370    /
 §\textcolor{lightgray}{3 7D}§  §\textcolor{lightgray}{AD AD B0 37 31 B6 34 31}§         .}--071641
 §\textcolor{lightgray}{2 7D}§  §\textcolor{lightgray}{57 CD CB CB DA C2 C1 C1}§         .}WMKKZBAA
 §\textcolor{lightgray}{1 7D}§  §\textcolor{lightgray}{0D 8A 31 B6 34 31 20 34}§         .}..1641 4
 §\textcolor{lightgray}{0 7D}§  §\textcolor{lightgray}{B9 32 83 B8 7E 7F 00 00}§         .}92.8....
\end{lstlisting}

\begin{center}
\begin{tabular}{p{1.5cm}p{2cm}p{8cm}}
\hfill\texttt{75008F}&source&\acrshort{AES} \acrshort{ICAO} 24-bit address for \acrshort{MRO}\\
\hfill\texttt{C5}&destination&\acrshort{GES} 305 at Perth\\
\hline
\hfill\textcolor{gray}{\texttt{\ldots}}&\textcolor{gray}{message}&\textcolor{gray}{[see above]}\\
\end{tabular}
\end{center}

\clearpage
\subsection{SATCOM Signalling Unit type}

Each \acrlong{SU} is prefixed by its type.  Because the first part
contains the address details and length information it is special, and is given a different \acrlong{SU} type
to the remainder of the parts:
\vfill
\begin{lstlisting}[language=SATCOM,numbers=left]
71 §\textcolor{lightgray}{75{ }{ }00 8F C5}§ §\textcolor{lightgray}{7D 09 60 FF FF}§         qu...}.`..
C8 §\textcolor{lightgray}{7D}§  §\textcolor{lightgray}{01 32 AE B9 CD AD CD 52}§         .}.2.9M-MR
C7 §\textcolor{lightgray}{7D}§  §\textcolor{lightgray}{4F 15 31 32 32 02 CD B0}§         .}O.122.M0
C6 §\textcolor{lightgray}{7D}§  §\textcolor{lightgray}{32 C1 CD C8 B0 B3 37 B0}§         .}2AMH0370
C5 §\textcolor{lightgray}{7D}§  §\textcolor{lightgray}{4F 46 46 B0 31 CD C1 D3}§         .}OFF01MAS
C4 §\textcolor{lightgray}{7D}§  §\textcolor{lightgray}{B3 37 B0 20 20 20 20 2F}§         .}370    /
C3 §\textcolor{lightgray}{7D}§  §\textcolor{lightgray}{AD AD B0 37 31 B6 34 31}§         .}--071641
C2 §\textcolor{lightgray}{7D}§  §\textcolor{lightgray}{57 CD CB CB DA C2 C1 C1}§         .}WMKKZBAA
C1 §\textcolor{lightgray}{7D}§  §\textcolor{lightgray}{0D 8A 31 B6 34 31 20 34}§         .}..1641 4
C0 §\textcolor{lightgray}{7D}§  §\textcolor{lightgray}{B9 32 83 B8 7E 7F 00 00}§         .}92.8....
\end{lstlisting}

The parts can now be transmitted in any order because each part contains its group identifier as well as 
its position within the group.  The type values are:

\begin{center}
\begin{tabular}{p{1.5cm}p{2cm}p{8cm}}
\hfill\texttt{71}&type&\acrlong{ISU}\\
\hfill\texttt{C8--C0}&type&Subsequent \acrlong{SU} (with sequence)\\
\end{tabular}
\end{center}

\subsection{SATCOM Signalling Unit checksum}

To ensure integrity for every individual part, an individual checksum is calculated
for each \acrfull{SU} and included to make each \acrlong{SU} an equal length of
twelve octets.
\vfill
\begin{lstlisting}[language=SATCOM,numbers=none]
§\textcolor{lightgray}{71 \textcolor{red}{75}}§  §\textcolor{lightgray}{\textcolor{red}{00 8F} C5 7D 09 60 FF FF}§  19 15  qu...}.`....
§\textcolor{lightgray}{C8 7D}§  §\textcolor{lightgray}{01 32 AE B\textcolor{red}{9} CD AD CD 52}§  B5 87  .}.2.9M-MR..
§\textcolor{lightgray}{C7 7D}§  §\textcolor{lightgray}{4F 15 31 32 32 02 CD B0}§  F2 28  .}O.122.M0.(
§\textcolor{lightgray}{C6 7D}§  §\textcolor{lightgray}{32 C1 CD C8 B\textcolor{red}{0} B\textcolor{red}{3} 3\textcolor{red}{7} B\textcolor{red}{0}}§  72 CA  .}2AMH0370r.
§\textcolor{lightgray}{C5 7D}§  §\textcolor{lightgray}{4F 46 46 B0 31 CD C1 D3}§  BD 72  .}OFF01MAS.r
§\textcolor{lightgray}{C4 7D}§  §\textcolor{lightgray}{B\textcolor{red}{3} 3\textcolor{red}{7} B\textcolor{red}{0} 20 20 20 20 2F}§  D0 AA  .}370    /..
§\textcolor{lightgray}{C3 7D}§  §\textcolor{lightgray}{AD AD B\textcolor{blue}{0} 3\textcolor{blue}{7} 3\textcolor{blue}{1} B\textcolor{blue}{6} 3\textcolor{blue}{4} 3\textcolor{blue}{1}}§  FB D0  .}--071641..
§\textcolor{lightgray}{C2 7D}§  §\textcolor{lightgray}{57 CD CB CB DA C2 C1 C1}§  BA 5F  .}WMKKZBAA._
§\textcolor{lightgray}{C1 7D}§  §\textcolor{lightgray}{0D 8A 3\textcolor{blue}{1} B\textcolor{blue}{6} 3\textcolor{blue}{4} 3\textcolor{blue}{1} 20 3\textcolor{green}{4}}§  46 9A  .}..1641 4F.
§\textcolor{lightgray}{C0 7D}§  §\textcolor{lightgray}{B\textcolor{green}{9} 3\textcolor{green}{2} 83 B8 7E 7F 00 00}§  12 2B  .}92.8.....+
\end{lstlisting}

For the Satcom error checking, a \acrlong{CRC16} is also used to produce the checksum
contained within each \acrlong{SU}.  The particular \gls{CRC16} variant used is called ``CCITT'' or ``X.25''
and starts with the value 65,535 (\texttt{FF\,FF}), before adding and multiplying the message values together.

\begin{center}
\begin{tabular}{p{1.5cm}p{2cm}p{8cm}}
\hfill\textcolor{gray}{\texttt{\ldots}}&\textcolor{gray}{message}&\textcolor{gray}{[split parts from above]}\\
\hline
\hfill\texttt{NN--NN}&checksum&calculated total of octets in each line\\
\end{tabular}
\end{center}

Even with all the wrapping that has taken place to encapsulate and
prepare the message for transmission, we can still pick out the aircraft
address, registration, callsign and \textcolor{red}{flight}
numbers; plus the date/\textcolor{blue}{time}, the \textcolor{green}{fuel}
load.  The Signalling Unit Set is now ready for transmission.

\section{Space segment}

{An \acrfull{AES} on-board the aircraft must follow the directions of the
\acrfull{GES} about when it can speak or transmit.\footnote{Several additional layers of
      wrapping, error correction checksums, interleaving and
      prepending of training sentences and the addition of the
      $\mathtt{E15AE893}_{16}$ magic word occur prior to T-Channel RF
      transmission.  These lower-level processes are not reflected in
      the (higher-level) unredacted \acrlong{SU} logfiles and are
      intentionally omitted within this briefing for the sake of
      readability: the full specification is
      \href{http://www.icao.int/safety/acp/Inactive\%20working\%20groups\%20library/ACP-WG-M-Iridium-8/IRD-SWG08-WP07\%20-\%20Old_AMSS_material_Ch.4_plus_Attachment.doc}{Part
        III of the AMS(R)S manual}.}

% https://github.com/sladen/inmarsat-9m-mro/blob/master/inmarsat-su-log-redacted.csv#L419

At 16:41:53.405 the \gls{GES} sent permission for the
%\acrshort{75008F}
\gls{AES}
%\gls{SDU}
on-board \gls{MRO} to be allowed
to start transmitting in the next superframe.  This frame
was scheduled to begin a few seconds later at 16:41:57.  The permitted frequency
for transmission was given as a nominal 1646.6850\,MHz, with a speed of 1,200
bits-per-seconds.

% https://github.com/sladen/inmarsat-9m-mro/blob/master/inmarsat-su-log-redacted.csv#L420

%3615.1850

%16:41:57.248
%16:41:57.409
%16:41:57.569 
%16:41:57.728 % Access request
%16:41:57.905
%16:41:58.048
%16:41:58.208
%16:41:58.399
%16:41:58.528
%16:41:58.697
%16:41:58.729
%m = (.248,.409,.569,.728,.905,1.048,1.208,1.399,1.528,1.697,1.729)
%>>> for a,b in zip(m[:-1],m[1:]): print b-a

The transmitted signals left via one of three Satcom L-Band antennas
on-board the aircraft\footnote{Two side-mounted High-Gain Antennas
  (HGA) and one roof-mounted Low-Gain Antenna (LGA).} and
$\frac{1}{8}$~second later were picked up by the \gls{3F1} in orbit, and relayed out again.
The frequency converter on-board the spacecraft increases the frequencies, changing them from L--Band to C--Band (even higher).  The L$\rightarrow$C frequency converter increases the frequency of all transmission signals by the same amount, so an incoming signal at
1646.6850\,MHz is rebroadcast at 3615.1850\,MHz.  The spacecraft
operates as a ``bent-pipe'' and is not aware
of the contents of the signals it is rebroadcasting.

\begin{figure}[]
%\setlength{\abovecaptionskip}{-0em plus 0.5em minus 0.5em}
\centering
% $ wget http://www.hq.nasa.gov/office/pao/History/presrep96/Presrp96/inmarbig.gif
% $ TZ=UTC ls --full-time inmarbig.gif 
% $ 1997-10-15 16:00:00
% convert inmarbig.gif -compress Zip inmarsat-3.pdf
\includegraphics[angle=180,width=0.5\textwidth]{{inmarsat-clipped}.pdf}
\caption[Inmarsat-3 spacecraft---15 October 1997. {[\href{http://www.hq.nasa.gov/office/pao/History/presrep96/Presrp96/inmarbig.gif}{original}]}]{Inmarsat-3 spacecraft. (NASA \textsc{public domain})}
\end{figure}
\vspace{-1em}

\section{Ground Earth Station}

A further $\frac{1}{8}$~second later again the rebroadcast signals began to
reach the \acrlongpl{GES} at Eik in Norway and Perth in Australia.
Additional approximate Doppler compensations were made to the incoming
signals from the spacecraft, the signal was down-converted
and demodulated, and ultimately the Signalling Units were decoded.
One \acrshort{SU} arrived every \textasciitilde160 milliseconds, with them being
logged to the \textbf{Signalling Unit logfiles} as
they went. It took 1.5~seconds for all ten \glsplural{SU} to arrive.

The information that is available to be logged can originate from
different categories. The first are external observations of the
transmission: the time, and frequency.  Second are the house keeping
information such as the Ocean Region that particular \gls{GES} is handling.  Third are
measurements taken while attempting to decode the transmission such as
the level of errors, received signal strength, and the frequency and
timing offsets plus a record of which piece of equipment performed the
decoding.

Fourth is the data itself, in its hexadecimal form.  Fifth is duplication and
highlighting of some of the received data for the ease of human
reading.  Those fields duplicated for human reading include the source and
destination address, the \gls{SU} type field, priority and reference number;
and the result of matching the address to a database to obtain the
\gls{ACARS} network provider (SITA) to replace the \texttt{FF\,FF} ``envelope'' for onwards delivery.

\subsection{Signalling Unit log file comparisons}

Figure \ref{164157.248-hex} and \ref{164157.248-photograph} match the earlier
\gls{OOOI} take-off message to the \gls{CSV} print outs shown to CNN by
Inmarsat on 23 May 2014 in the pages of unredacted \gls{SU} log
files.  In Figure \ref{164157.248-photograph} the 3615.1850\,MHz
frequency\footnote{Nominal channel frequency. Doppler pre- and post-compensation discussion is omitted in this briefing note.}
 can be seen as $\mathtt{36FA}$ in the \emph{Channel Name} column.\footnote{
\,$\mathtt{hex}(\frac{{3615.1850\,{MHz}\ input}-{1968.5\,MHz\ spacecraft\_local\_oscillator_{LtoC}}-{101.5\,MHz\ tx\_shift_{LtoL}}-{1510.0\,MHz\ base}}{0.0025\,MHz {\ channel\_width}})$}

%\vfill
\begin{figure}[ht!]%
\setlength{\abovecaptionskip}{0.5em plus 0.5em minus 0.5em}%
\label{164157.248-hex}%
\begin{lstlisting}[language=SATCOM,numbers=left]
71 §\textcolor{red}{75}§  §\textcolor{red}{00 8F}§ C5 7D 09 60 FF FF  19 15  qu...}.`....
C8 7D  01 32 AE B§\textcolor{red}{9}§ CD AD CD 52  B5 87  .}.2.§\textcolor{red}{9}§M-MR..
C7 7D  4F 15 31 32 32 02 CD B0  F2 28  .}O.122.M0.(
C6 7D  32 C1 CD C8 B§\textcolor{red}{0}§ B§\textcolor{red}{3}§ 3§\textcolor{red}{7}§ B§\textcolor{red}{0}§  72 CA  .}2AMH§\textcolor{red}{0370}§r.
C5 7D  4F 46 46 B0 31 CD C1 D3  BD 72  .}§{\setlength{\fboxsep}{0.15em}\colorbox{lightskyblue}{OFF01MAS}}§.r
C4 7D  B§\textcolor{red}{3}§ §3\textcolor{red}{7}§ §B\textcolor{red}{0}§ 20 20 20 20 2F  D0 AA  .}§{\setlength{\fboxsep}{0.15em}\colorbox{lightskyblue}{\textcolor{red}{370}{ }{ }{ }{ }/}}§..
C3 7D  AD AD B§\textcolor{blue}{0}§ 3§\textcolor{blue}{7}§ 3§\textcolor{blue}{1}§ B§\textcolor{blue}{6}§ 3§\textcolor{blue}{4}§ 3§\textcolor{blue}{1}§  FB D0  .}§{\setlength{\fboxsep}{0.15em}\colorbox{lightskyblue}{{-}{-}\textcolor{blue}{071641}}}§..
C2 7D  57 CD CB CB DA C2 C1 C1  BA 5F  .}§{\setlength{\fboxsep}{0.15em}\colorbox{lightskyblue}{WMKKZBAA}}§._
C1 7D  0D 8A 3§\textcolor{blue}{1}§ B§\textcolor{blue}{6}§ 3§\textcolor{blue}{4}§ 3§\textcolor{blue}{1}§ 20 3§\textcolor{green}{4}§  46 9A  .}§{\setlength{\fboxsep}{0.15em}\colorbox{lightskyblue}{..\textcolor{blue}{1641} \textcolor{green}{4}}}§F.
C0 7D  B§\textcolor{green}{9}§ 3§\textcolor{green}{2}§ 83 B8 7E 7F 00 00  12 2B  .}§{\setlength{\fboxsep}{0.15em}\colorbox{lightskyblue}{\textcolor{green}{92}}}§.8.....+
\end{lstlisting}
\caption[Unredacted SUs 16:41:57{\small.248}--16:41:58{\small.729} (OOOI SU user data)---7 March 2014]{Unredacted \acrlongpl{SU} 16:41:57{\small.248} to 16:41:58{\small.729}. Highlighted numbers are flight/\textcolor{red}{callsign}, date/\textcolor{blue}{time}, \textcolor{green}{fuel}. Lines start on the \textsc{Left} (\texttt{71}, \texttt{C8}, \texttt{C7}, \ldots, \texttt{C0}).  Compare to Figure~\ref{164157.248-photograph}.}%
\end{figure}

\begin{figure}[ht!]%
\centering%
\label{164157.248-photograph}%
\setlength{\abovecaptionskip}{0.5em plus 0.5em minus 0.5em}%
\setlength{\fboxrule}{0.05mm}
\resizebox{0.90\textwidth}{!}{%
%\fcolorbox{black}{white}{%
%36 38 27 32
%452mm-101mm-94mm,289mm-78mm-75mm
%\clipbox{101mm 78mm 94mm 75mm}{
% left bottom right top
%\clipbox{37.5mm 30mm 35mm 29mm}{
\clipbox{48mm -32mm 33mm 122mm}{%
\begin{picture}(128.0mm, 128.0mm)(0,0)%
\put(0,0){\includegraphics[angle=-39,width=128mm,keepaspectratio]{page1-off.jpg}}%
\put(0,0){\includegraphics[angle=-39,width=128mm,keepaspectratio]{{page1-oooi-overlay-minus-thumb}.pdf}}%
\end{picture}%
}}%}%
\caption[Unredacted SUs 16:41:57{\small.248}--16:41:58{\small.729} (OOOI annotated photo)---23 May 2014]{Unredacted \acrlong{SU} log 16:41:57{\small.248} to 16:41:58{\small.729}. Highlighted numbers are flight/\textcolor{red}{callsign}, date/\textcolor{blue}{time}, and \textcolor{green}{fuel}.  Cross-reference with values in Figure~\ref{164157.248-hex}.  Highlighted hexadecimal user data lines start on the \textsc{Right} (\texttt{71}, \texttt{C8}, \texttt{C7}, \ldots, \texttt{C0}) and wrap to \textsc{Left} side.  \hfill(CNN)}%
\end{figure}

\begin{figure}[]%
%\setlength{\abovecaptionskip}{-0em plus 0.5em minus 0.5em}
\centering%
\label{164223.039-photograph}%
%\includegraphics[angle=-4,width=0.8\textwidth]{{inmarsat-page3-overlay}.pdf}
\resizebox{0.94\textwidth}{!}{%
%\fcolorbox{black}{white}{%
\includegraphics[angle=-4,width=0.94\textwidth]{{inmarsat-page3-overlay-rotated-text}.pdf}}%
%}%
\caption[Unredacted SUs 16:42:23{\small.039}--16:42:25{\small.909}---23 May 2014]{Unredacted \acrlong{SU} log 16:42:23{\small.039}--16:42:25{\small.909}. Examples of fields that are present (Time, BFO, BTO, \dots) in Figure~\ref{164223.039-redacted-page} verses other fields not present in the later Malayasian publication (Framing, Sequence, Received Signal Strength, \ldots)   \hfill(CNN)}%
\end{figure}

\begin{figure}[]%
%\setlength{\abovecaptionskip}{-0em plus 0.5em minus 0.5em}
\label{164223.039-redacted-page}%
\centering%
\resizebox{0.94\textwidth}{!}{%
{\setlength{\fboxsep}{3mm}%
\fcolorbox{black}{white}{%
%\clipbox{19mm 15mm 33mm 126.2mm}{%
\clipbox{19mm 0mm 33mm 126.2mm}{%
%\includegraphics[angle=-0.7,page=21,width=0.8\textwidth,trim=19mm 15mm 33mm 126mm,clip]{{"../MH370 Data Communication Logs"}.pdf}}
\includegraphics[angle=0,page=21]{{"../MH370 Data Communication Logs"}.pdf}}}}}%
\caption[\hspace{0.96em}Redacted SUs 16:42:23{\small.039}--16:42:25{\small.909}---27 May 2014]{Redacted \acrlong{SU} log 16:42:23{\small.039}--16:42:25{\small.909} from the Malayasian publication. 
Artifacts of manual preparation are apparent in the stepped misalignment of the first column verses remaining columns---visible at the cut-line.
Match values with Figure~\ref{164223.039-photograph}. \hfill(\gls{MDCA})}%
\end{figure}%
\vspace{-1em}

\vspace{-1em}

\vspace{0.25em}
\begin{figure}[ht!]%
\label{cnn-dickinson-field-headings}
%\setlength{\abovecaptionskip}{-0.2em plus 0.5em minus 0.5em}
\centering%
\resizebox{0.94\textwidth}{!}{%
% left bottom right top
\clipbox{0 2.55mm 0 0}{%
\begin{picture}(128.0mm, 72.0mm)(0,0)%
\put(0,0){\includegraphics[width=128.0mm,keepaspectratio]{cnn-dickinson-page2-example-1280x720.jpg}}%
\put(0,0){\includegraphics[width=128.0mm,keepaspectratio]{{cnn-dickinson-page2-example-overlay}.pdf}}%
\end{picture}%
}}%
\caption[Mark Dickinson with 16:42:04{\small.408} example---23 May 2014]{%
Mark Dickinson showing various unredacted log file sections to CNN on 23~May 2014.\\
\textsc{Centre:} 16:42:04{\small.408} \acrshort{SU} highlighted.
\textsc{Left:} Mark Dickinson, at Inmarsat HQ. \hfill(CNN)}
\end{figure}
\vspace{-0.25em}

\begin{figure}[]%
%\setlength{\abovecaptionskip}{-0em plus 0.5em minus 0.5em}
\label{164204.408-page1}%
\centering%
\resizebox{0.94\textwidth}{!}{%
{\setlength{\fboxsep}{2mm}%
\fcolorbox{black}{white}{%
%left bottom right top
%\clipbox{28mm 71.6mm 25mm 98.58mm }{%
\clipbox{27.6mm 71.6mm 34mm 98.58mm }{%
%\includegraphics[angle=-0.7,page=21,width=0.8\textwidth,trim=19mm 15mm 33mm 126mm,clip]{{"../MH370 Data Communication Logs"}.pdf}}
\includegraphics[angle=0,page=1]{{"../MH370 Data Communication Logs"}.pdf}}}}}%
% Why can't I use \phantom{s} here?
\caption[Unredacted SU\hspace{0.37em} 16:42:04{\small.408} (MDCA example)---27 May 2014]{Later \acrlong{MDCA} publication with 16:42:04{\small.408} example given on Page~1 only.  Cross-checked for validity with previously obtained
Figure~\ref{164204.408-photograph}. \hfill(\gls{MDCA})}%
\end{figure}

\begin{figure}[]%
\vspace{3em}%
%\setlength{\abovecaptionskip}{-0em plus 0.5em minus 0.5em}
\label{164204.408-photograph}
\centering%
\resizebox{0.94\textwidth}{!}{%
\begin{picture}(173.0mm, 70.4mm)(0,0)%
\put(0,0){%
\fcolorbox{black}{white}{%
\includegraphics[width=173.0mm,keepaspectratio]{page2-extract-alone.pdf}}%
}%
\put(113mm,50mm){%
\fcolorbox{black}{white}{%
\resizebox{!}{40mm}{%
\begin{picture}(128.0mm, 72.0mm)(0,0)%
\put(0,0){\includegraphics[width=128.0mm,keepaspectratio]{cnn-dickinson-page2-example-inlay-1280x720.jpg}}%
\put(0,0){\includegraphics[width=128.0mm,keepaspectratio]{{cnn-dickinson-page2-example-inlay-overlay}.pdf}}%
\end{picture}%
}}}%
\end{picture}%
}%
%\includegraphics[angle=-0.7,page=21,width=0.8\textwidth,trim=19mm 15mm 33mm 126mm,clip]{{"../MH370 Data Communication Logs"}.pdf}}
%\includegraphics[]{{extract-page1-field-headings}.pdf}}%
\caption[Unredacted SU\hspace{0.37em} 16:42:04{\small.408} (Inmarsat)---23 May 2014]{Unredacted \acrlong{SU} log example with 16:42:04{\small.408} highlighted.  Only those fields highlighted are present in the main body of the later Malayasian release.
%Compare match with Figure~\ref{headings-photograph}
\hfill(CNN)}%
\end{figure}


%\vfill
%\vspace{-0.25em}
\vspace{0.25em}
\begin{figure}[ht!]%
\label{cnn-dickinson-field-headings}
%\setlength{\abovecaptionskip}{-0.2em plus 0.5em minus 0.5em}
\centering%
\resizebox{0.94\textwidth}{!}{%
% left bottom right top
\clipbox{0 2.55mm 0 0}{%
\begin{picture}(128.0mm, 72.0mm)(0,0)%
\put(0,0){\includegraphics[width=128.0mm,keepaspectratio]{cnn-dickinson-page1-headings-1280x720.jpg}}%
\put(0,0){\includegraphics[width=128.0mm,keepaspectratio]{{cnn-dickinson-page1-headings-overlay}.pdf}}%
\end{picture}%
}}%
\caption[Mark Dickinson with log field-headings---23 May 2014]{%
Mark Dickinson showing log field-headings to CNN on 23~May 2014.\\
\textsc{Left:} Unredacted field headings. 
\textsc{Right:} Mark Dickinson, at Inmarsat HQ. \hfill(CNN)}
\end{figure}
\vspace{-0.25em}


\begin{figure}[]%
%\setlength{\abovecaptionskip}{-0em plus 0.5em minus 0.5em}
\label{field-headings-redacted-page}%
\centering%
\resizebox{0.94\textwidth}{!}{%
{\setlength{\fboxsep}{2mm}%
\fcolorbox{black}{white}{%
%left bottom right top
\clipbox{28mm 103.5mm 25mm 70mm}{%
%\includegraphics[angle=-0.7,page=21,width=0.8\textwidth,trim=19mm 15mm 33mm 126mm,clip]{{"../MH370 Data Communication Logs"}.pdf}}
\includegraphics[angle=0,page=1]{{"../MH370 Data Communication Logs"}.pdf}}}}}%
\caption[Unredacted SU log field headings (MDCA example)---27 May 2014]{Later \acrlong{MDCA} publication with unredacted example headings on Page~1 only.  Cross-checked for validity with previously obtained
Figure~\ref{field-headings-unredacted-page1}. \hfill(\gls{MDCA})}%
\end{figure}

\begin{figure}[]%
\vspace{3em}%
%\setlength{\abovecaptionskip}{-0em plus 0.5em minus 0.5em}
\label{field-headings-unredacted-page1}
\centering%
\resizebox{0.94\textwidth}{!}{%
\begin{picture}(173.0mm, 70.4mm)(0,0)%
\put(0,0){%
\fcolorbox{black}{white}{%
\includegraphics[width=173.0mm,keepaspectratio]{{extract-page1-field-headings}.pdf}}%
}%
\put(113mm,50mm){%
\fcolorbox{black}{white}{%
\resizebox{!}{40mm}{%
\begin{picture}(128.0mm, 72.0mm)(0,0)%
\put(0,0){\includegraphics[width=128.0mm,keepaspectratio]{cnn-dickinson-page1-headings-inlay.jpg}}%
\put(0,0){\includegraphics[width=128.0mm,keepaspectratio]{{cnn-dickinson-page1-headings-inlay-overlay}.pdf}}%
\end{picture}%
}}}%
\end{picture}%
}%
%\includegraphics[angle=-0.7,page=21,width=0.8\textwidth,trim=19mm 15mm 33mm 126mm,clip]{{"../MH370 Data Communication Logs"}.pdf}}
%\includegraphics[]{{extract-page1-field-headings}.pdf}}%
\caption[Unredacted SU log field headings (Inmarsat)---23 May 2014]{Unredacted \acrlong{SU} log field-headings.  Only those highlighted fields were present in the main body of the later Malayasian release.
%Compare match with Figure~\ref{headings-photograph}
\hfill(CNN)}%
\end{figure}



%%%%%%%%%%%%%%%%%%%%%%%%%%%%%%%%%%%%%%%%%%%%%%%%%%%%%%%%%%%%%%%%%%%%%%%%%%%%%%%%%%%%%%%%%%%%%%%


%\glsaddall
%\listoffigures
\clearpage
\section*{Conclusion}

We have traced how a simple \gls{OOOI} take-off message is wrapped up
in preparation for transmission by \gls{ACARS} and Satcom.  We have
identified this complete message in the logged \acrlongpl{SU} of the
unredacted logs publicised by Inmarsat
(Figure~\ref{164157.248-photograph}).  We have observed that the same
content is not present in the later \acrlong{MDCA} publication and deduced
redaction of 19 out of 28 columns in the later Malayasian publication
of 27 May 2014.

In line with other safety-of-life systems, \emph{Classic Aero} Satcom
messages are transmitted in the clear and by their nature are
rebroadcast to one-third of the planet in both directions.  Such
communications can not be in anyway be regarded as
``secret'' or ``confidential''\,\footnote{With regard to \gls{3F1} the well-known
  106.5\textdegree\ azimuth of the antenn\ae\ at \acrshort{GCHQ} Bude
  is tangentially noted.}\,and this appears to have been reflected in
Inmarsat's actions and statements.

\printglossary[type=\acronymtype,title=List of Terms,toctitle=Terms and abbreviations]

\vfill
\raggedleft{\small\parbox{10em}{\textsc{except where\\otherwise attributed}\hspace{1em}}\raisebox{-0.65em}{\includegraphics[height=2em]{cc-by.pdf}}}
\clearpage
\tableofcontents
\renewcommand*\listtablename{Tables, Figures \& Datasets}
\listoftableandfigures
%\listoffigures
%\listoftables

\vfill

\setlength{\footnotemargin}{0.5em}
%\subsubsection*{Postscript}
Multiple frames and angles were used for dewarping.
Data values found obscured or unreadable were recovered using the
two-layers of CRC16 checksums present (Kermit\,\&\,X.25)\footnote{
\begin{tabular}{ccl}%
\small{\acrshort{ACARS}}&\small{``Kermit''}&\tiny{\texttt{crc\_algorithms.Crc(width=16, poly=0x1021, reflect\_in=True, xor\_in=0x0000, reflect\_out=True, xor\_out=0x0000)}}\\
\small{\acrshort{AMSS}}&\small{``X.25''}&\tiny{\texttt{crc\_algorithms.Crc(width=16, poly=0x1021, reflect\_in=True, xor\_in=0xffff, reflect\_out=True, xor\_out=0xffff)}}%
\end{tabular}}
to facilitate
user-data error correction and recovery, in a similiar manner to solving a giant Sudoko
grid.\unskip\parfillskip-0.08em \par\vspace{-0.4em}

\end{document}




